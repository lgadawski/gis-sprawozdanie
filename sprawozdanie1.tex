\documentclass[10pt,a4paper]{article}
\usepackage[utf8]{inputenc}
\usepackage{polski}
\usepackage{amsmath}
\usepackage{amsfonts}
\usepackage{amssymb}
\usepackage{graphicx}
\usepackage{verbatim}
\author{Korzeniowski Wojciech, Gadawski Łukasz}
\title{Wyznaczenie sumarycznie najkrótszych ścieżek rozłącznych krawędziowo}
\begin{document}
\maketitle

\section{Opis zadania}
Realizacja projektu będzie polegała na zaimplementowaniu zmodyfikowanej wersji algorytmu Dijkstry, która umożliwi znalezienie dwóch rozłącznych krawędziowo ścieżek w skierowanym grafie dla dowolnej pary wierzchołków.

\subsection{Wykorzystany algorytm}
Na podstawie \textit{Appendix A} do artykułu \cite{telecom}.
\begin{enumerate}
\item Rozpocznij z:

$$
d(A) = 0, 
$$

\[d(i) = \left\{
  \begin{array}{lr}
    l(A, i) & dla i \in N(A),\\
    \infty & dla i \notin N(A)
  \end{array}
\right.
\]
$$
S = V - \{A\}
$$
$$
\forall P(i) = A, i \in S
$$

\item Krok:

\begin{itemize}
\item[-] znajdź $j : d(j) = min d(i), i \in S$,
\item[-] $S = V - \{j\}$,
\item[-] jeśli $j = Z$ to STOP, w p.p. idź do kroku 3.
\end{itemize}

\item Krok:
$$
\forall d(j) + l(j, i) < d(i), \Rightarrow d(i) = d(j) + l(j, i), P(i) = j
$$
$$
S = S \cup \{i\}
$$
idź do kroku 2

\end{enumerate}

\subsection{Założenia realizacyjne}
Implementacja projektu zostanie wykonana w języku Java w formie aplikacji konsolowej. Dane wejściowe dotyczące rozpatrywanego grafu zostaną wczytane z pliku testowego o następującej strukturze:

\verbatiminput{simpleGraph.txt}

gdzie komentarz rozpoczyna się od znaku \#, każda linia reprezentuję pojedynczą krawędź rozpatrywanego grafu przy czym pierwsza kolumna opisuję etykietę wierzchołka początkowego danej krawędzi, kolumna druga definiuje etykietę wierzchołka końcowego, a w ostatniej kolumnie znajduje się waga konkretnej krawędzi. Wynik działania algorytmu będą reprezentowały listy etykiet wierzchołków najkrótszych ścieżek jeśli takie będą istniały oraz wagi konkretnych ścieżek.

\begin{thebibliography}{9}
\bibitem{telecom} Ramesh Bhandari \emph{Optimal Diverse Routing in Telecommunication Fiber Networks}
%\bibitem{Cormen} T.H. Cormen, C.E. Leiserson, R.L. Rivest, \emph{Wprowadzenie do algorytmów}, Wydawnictwo
Naukowo-Techniczne, Warszawa 2001.
\end{thebibliography}

\end{document}