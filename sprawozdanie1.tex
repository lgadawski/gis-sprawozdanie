\documentclass[10pt,a4paper]{article}
\usepackage[utf8]{inputenc}
\usepackage{polski}
\usepackage{amsmath}
\usepackage{amsfonts}
\usepackage{amssymb}
\usepackage{graphicx}
\usepackage{verbatim}
\author{Korzeniowski Wojciech, Gadawski Łukasz}
\title{Wyznaczenie sumarycznie najkrótszych ścieżek rozłącznych krawędziowo}
\begin{document}
\maketitle

\section{Opis zadania}
Realizacja projektu będzie polegała na zaimplementowaniu zmodyfikowanej wersji algorytmu Dijkstry, która umożliwi znalezienie dwóch rozłącznych krawędziowo ścieżek w skierowanym grafie dla dowolnej pary wierzchołków.

\subsection{Założenia realizacyjne}
Implementacja projektu zostanie wykonana w języku Java w formie aplikacji konsolowej. Dane wejściowe dotyczące rozpatrywanego grafu zostaną wczytane z pliku testowego o następującej strukturze:

\verbatiminput{simpleGraph.txt}

gdzie rozpoczyna się od znaku \#, każda linia reprezentuję pojedynczą krawędź rozpatrywanego grafu przy czym pierwsza kolumna opisuję etykietę wierzchołka początkowego danej krawędzi, kolumna druga definiuje etykietę wierzchołka końcowego, a w ostatniej kolumnie znajduje się waga konkretnej krawędzi.

\end{document}